% !TEX program = xelatex

\documentclass{resume}
\usepackage{comment}
\usepackage{multicol}
\usepackage{hyperref} %\hypersetup{colorlinks=false}
%\usepackage{wrapfig}
%\usepackage{zh_CN-Adobefonts_external} % Simplified Chinese Support using external fonts (./fonts/zh_CN-Adobe/)
%\usepackage{zh_CN-Adobefonts_internal} % Simplified Chinese Support using system fonts
 
\begin{document}
\pagenumbering{gobble} % suppress displaying page number

\name{Oswin Rodrigues}
\program{\faRocket\ 2B Mechatronics Engineering}
%\focus{Hardware}{Embedded}

% {E-mail}{mobilephone}{homepage}
\contactInfo{orodrigues@uwaterloo.ca}{+1-226-606-6220}{oswinrodrigues}
% {E-mail}{mobilephone}
%\basicContactInfo{xxx@yuanbin.me}{(+86) 131-221-87xxx}

\section{\faWrench\ Tools}

\begin{itemize}[parsep=0.5ex]
  \item %Code: 
  C\textperiodcentered 
  C++\textperiodcentered
  Python\textperiodcentered
  Ladder Logic\textperiodcentered
  JavaScript\textperiodcentered
  MATLAB\textperiodcentered
  Assembly\textperiodcentered
  VHDL
  \item %Circuit:
  Circuit Design\textperiodcentered
  Arduino\textperiodcentered
  Soldering \& Rework\textperiodcentered
  Upverter\textperiodcentered
  EAGLE\textperiodcentered
  Multimeter \& Oscilloscope
  %\item %3D CAD Experience:
  %AutoCAD\textperiodcentered
  %Solidworks (Entire work-term, not mentioned below, spent on 3D/2D CAD during Fall 2014)
\end{itemize}

\section{\faLineChart\ Experience}

\datedsubsection{\textbf{EDA/CAD Engineer Intern}}{May -- Aug 2015}
\company{\textbf{Upverter Inc.}}{Toronto, ON}
\textit{Enhancing PCB CAD tool features in software (JavaScript, Python) and hardware avenues.}
\begin{itemize}
  \item Created and verified symbols and footprints for 150+ electronic components.
  \item Implemented component-tagging feature, using pin names for functionality.
  \item Adjusted prioritization of design rules in layout constraint manager.
  %\item Fixed incorrect drawing and positioning of constraint violation layout bodies.
  \item Refactored event-listening logic for drawing nets in schematic editor.
  \item Corrected pin manager's oversight in tracking connection mappings.
  \item Enabled efficient BGA row enumeration during footprint generation.
\end{itemize}

\datedsubsection{\textbf{Neuro-Robotics Lab Research Assistant}}{Feb -- Apr 2014}
\company{\textbf{University of Waterloo}}{Waterloo, ON}
\textit{Using ROS-run Turtlebot\textsuperscript{TM} for social navigation research purposes.}
\begin{itemize}
  \item Wrote C++ and Python nodes to implement basic navigation stack on Turtlebot. 
  \item Published sensor, odometry and transform messages to mobile base.
  \item Tweaked existing open-source code for advanced algorithms: person-detection, SLAM navigation.
  %\item Troubleshot various problems arising due to incomplete-but-improving familiarity with ROS.
\end{itemize}

\begin{comment}
\datedsubsection{\textbf{Mechanical Design Co-op}}{Sep -- Dec 2014}
\company{\textbf{Prodomax Automation Inc.}}{Barrie, ON}
\textit{Designing jigs and fixtures in Solidworks for automotive part-assembly stations.}
\begin{itemize}
  \item Modeled custom tooling in two stations for assembling a vehicle's seat track mechanism. One inserted an anti-collapse spacer and the other stamp-pressed a bushing.
  \item Detailed and ballooned numerous part and assembly drawings extensively.
\end{itemize}
\end{comment}

\section{\faFlask\ Projects}

\datedsubsection{\textbf{UW Robotics Team \& Waterloo Autonomous Vehicles Lab}}{Jan 2015 -- Present}
\begin{itemize}
  \item Modified EAGLE schematics and layouts for Arduino motor shield. 
  \item Soldered SMT and THT components onto multiple bare shields.
  \item Currently designing and implementing a wireless (RF) e-stop mechanism for racing robot.
  \item Currently rebuilding and parts-sourcing a Mars Rover's electrical box. 
\end{itemize}

\datedsubsection{\href{https://github.com/oswinrodrigues/LED-Matrix-And-Accelerometer}{\textbf{Tilt-Sensitive LED Matrix Panel}}}{Personal Project, Ongoing}
\textit{`Moving' a single lit LED on panel by physically tilting it. This uses:}
\begin{itemize}
  \item Arduino microcontroller for handling the `smarts and magic'.
  \item ADXL335 accelerometer for controlling the tilt functionality. 
  \item 74HC595N shift register (SIPO) for I/O expansion on the Arduino board.
\end{itemize}

\datedsubsection{\textbf{Hackathons}}{Various}
\begin{itemize}
  \item \href{https://github.com/DChang87/HtN}{Pebble-run dosage notification service - \textit{SmartMeds}; used C.} \hfill Hack the North, 2015
  \item \href{jkkd.com}{IMU-based instructor - \textit{Yoga Yoda}; developed business case. \hfill} PCH Hardware Hackathon, 2015
  \item \href{http://devpost.com/software/drumyo}{Myo-controlled air drum kit - \textit{DruMyo}; used C++. \hfill} hackWaterloo, 2014
  \item \href{https://github.com/fanwashere/SolidWorksControls}{Myo-enabled Solidworks controller; used Lua. \hfill} Hack the North, 2014
\end{itemize}

\begin{multicols}{2}

\section{\faGraduationCap\ Education}
\textbf{Mechatronics Engineering}\par
Candidate for BASc\par
2013 -- Present\par
Class of 2018\par
University of Waterloo, Waterloo, ON

\columnbreak

\section{\faBook\ Courses}
\href{http://www.ucalendar.uwaterloo.ca/1516/COURSE/course-MTE.html#MTE120}{Circuits \hfill 93\%} \par
\href{http://www.ucalendar.uwaterloo.ca/1516/COURSE/course-MTE.html#MTE262}{Data Structures \& Algorithms \hfill 94\%}\par
\href{http://www.ucalendar.uwaterloo.ca/1516/COURSE/course-MTE.html#MTE262}{Microprocessors \& Digital Logic \hfill 78\%}\par
\href{http://www.ucalendar.uwaterloo.ca/1516/COURSE/course-MTE.html#MTE220}{Sensors \& Instrumentation\hfill -----}\par
\href{http://www.ucalendar.uwaterloo.ca/1516/COURSE/course-MTE.html#MTE241}{Computer Structures \& Real-Time Systems\hfill -----}

\end{multicols}

\end{document}