% !TEX program = xelatex

\documentclass{resume}
\usepackage{comment}
\usepackage{multicol}
\usepackage{hyperref} %\hypersetup{colorlinks=false}
%\usepackage{wrapfig}
 
\begin{document}
\pagenumbering{gobble} % suppress displaying page number

\name{Oswin Rodrigues}
\program{Mechatronics Engineering, 3rd Year}
%\focus{Hardware}{Embedded Software}

% {E-mail}{mobilephone}{homepage}
\contactInfo{orodrigues@uwaterloo.ca}{+1 226 606 6220}{oswinrodrigues}

\section{\faTrophy\ Goals}

\begin{itemize}[parsep=0.5ex]
  \item To excel and grow in the design and development of \textbf{hardware} and \textbf{embedded software} systems.
\end{itemize}

\section{\faWrench\ Tools}

\begin{itemize}[parsep=0.5ex]
  \item %Circuit:
  Schematics \& Layouts\textperiodcentered
  Soldering \& Rework\textperiodcentered
  Multimeter \& Oscilloscope\textperiodcentered
  Arduino \& Raspberry Pi
  \hfill \faBolt

  \item %Code:
  C\textperiodcentered 
  C++\textperiodcentered
  Python\textperiodcentered
  MATLAB\textperiodcentered
  FPGA Programming\textperiodcentered
  Ladder Logic\textperiodcentered
  ROS\textperiodcentered
  JavaScript%\textperiodcentered
  %Assembly\textperiodcentered
  %VHDL\textperiodcentered
  \hfill \faCode

  %\item %3D CAD Experience:
  %AutoCAD\textperiodcentered
  %Solidworks (Entire work-term, not mentioned below, spent on 3D/2D CAD during Fall 2014)\textperiodcentered
\end{itemize}

\section{\faLineChart\ Experience}

\datedsubsection{\textbf{AI \& Robotics Engineer}}{Winter 2016}
\company{\textbf{Stealth-mode AI \& Robotics Startup}}{Toronto, ON}
\textit{Robot-wrangling with Python over distributed communication architecture.}
\begin{itemize}
  \item Sourced, tested and integrated components into system via custom-coded drivers.
  \item Soldered robots' power boards and executed safety bringup.
\end{itemize}
\begin{comment}
  Informal intro to ML / N-nets, FK / IK, Grasping, FSM, DB (ES)
  People skills: pair-programming, 1-on-1, standups, blog posts, Happy Hour, AI planning, CI for internal processes, HW concept review
  Python fu: rapid prototyping / scripting AND enterprise-level code --> classes, exception-handling, logging, PDB debugging, linting, documentation
\end{comment}

\datedsubsection{\textbf{EDA \& CAD Engineer}}{Summer 2015}
\company{\textbf{Upverter Inc.}}{Toronto, ON}
\textit{Enhancing PCB CAD features in hardware and software avenues.}
\begin{itemize}
    \item Created and verified symbols and footprints for 150+ electronic components.
    \item Used JavaScript to re-factor features and fire-fight bugs extensively.
  %\item Implemented component-tagging feature, using pin names to categorize functionality.  
\end{itemize}
\begin{comment}
  Adjusted prioritization of design rules in layout constraint manager.
  Fixed incorrect drawing and positioning of constraint violation layout bodies.
  Re-factored event-listening logic for drawing nets in schematic editor.
  Corrected pin manager's oversight in tracking connection mappings.
  Enabled efficient BGA row enumeration during footprint generation.
\end{comment}

\datedsubsection{\textbf{Junior Mechanical Designer}}{Fall 2014}
\company{\textbf{Prodomax Automation Inc.}}{Barrie, ON}
\textit{CAD-ing custom jigs and fixtures in Solidworks for automotive part-assembly stations.}
\begin{comment}
\begin{itemize}
  \item Modeled custom tooling in two assembly stations for a vehicle's seat track mechanism. % Inserted anti-collapse spacer, stamp-pressed bushing.
  \item Detailed and ballooned numerous part and assembly drawings.
\end{itemize}
\end{comment}

\datedsubsection{\textbf{Neuro-Robotics Lab Research Assistant}}{Winter 2014}
\company{\textbf{University of Waterloo}}{Waterloo, ON}
\textit{Implementing C++ and Python nodes on ROS-run Turtlebot for navigation research.}
\begin{comment}
\begin{itemize}
  \item Wrote C++ and Python nodes to implement navigation stack on Turtlebot. 
  \item Published sensor, odometry and transform messages to mobile base.
  \item Tweaked existing open-source code for advanced algorithms: person-detection, SLAM navigation.
  \item Gained immense troubleshooting experience associated with accommodating open-source software.
\end{itemize}
\end{comment}

\section{\faFlask\ Projects}

\datedsubsection{\textbf{UW Robotics Team \& Waterloo Autonomous Vehicles Lab}}{Various}%{\faCar}
\begin{itemize}
  \item Verified and modified EAGLE schematics and layouts for Arduino motor shield. % on racing robot car.
  \item Soldered SMT and THT components onto three bare PCBs and probed subsequently.
  %\item Researched design considerations for wireless (RF) e-stop mechanism on robot racing car.
\end{itemize}

\datedsubsection{\href{https://github.com/oswinrodrigues/LED-Matrix-And-Accelerometer}{\textbf{Tilt-Sensitive LED Matrix Panel}}}{Winter 2016}%{\faLightbulbO}
\begin{itemize}
  \item Wrote LED driver that uses two 74HC595N shift registers for I/O expansion.
  \item Wrote IMU driver for ADXL335 and MPU6050, with filter to integrate gyro and accelerometer.
\end{itemize}

\begin{comment}
\datedsubsection{\textbf{Hackathons}}{Various}
\begin{itemize}
  \item \href{https://github.com/DChang87/HtN}{Pebble-run dosage notification service - \textit{SmartMeds}; used C.} \hfill Hack the North, 2015
  \item \href{jkkd.com}{IMU-based instructor - \textit{Yoga Yoda}; developed business case. \hfill} PCH Hardware Hackathon, 2015
  \item \href{http://devpost.com/software/drumyo}{Myo-controlled air drum kit - \textit{DruMyo}; used C++. \hfill} hackWaterloo, 2014
  \item \href{https://github.com/fanwashere/SolidWorksControls}{Myo-enabled Solidworks controller; used Lua. \hfill} Hack the North, 2014
\end{itemize}
\end{comment}

\begin{multicols}{2}

\section{\faGraduationCap\ Education}
\textbf{Mechatronics Engineering, Honors, BASc.}\par
%Candidate for BASc\par
University of Waterloo, Waterloo, ON\par
%2013 -- Present\par
Class of 2018

\columnbreak

\section{\faBook\ Courses}

\href{http://www.ucalendar.uwaterloo.ca/1516/COURSE/course-MTE.html#MTE341}{Microprocessor Systems \& Interfacing \hfill 95\%} \par
\href{http://www.ucalendar.uwaterloo.ca/1516/COURSE/course-MTE.html#MTE220}{Sensors \& Instrumentation\hfill 80\%}\par
\href{http://www.ucalendar.uwaterloo.ca/1516/COURSE/course-MTE.html#MTE120}{Circuits \hfill 93\%}
%\href{http://www.ucalendar.uwaterloo.ca/1516/COURSE/course-MTE.html#MTE320}{Actuators \& Power Electronics\hfill 92\%}\par
%\href{http://www.ucalendar.uwaterloo.ca/1516/COURSE/course-MTE.html#MTE262}{Data Structures \& Algorithms \hfill 94\%}\par
%\href{http://www.ucalendar.uwaterloo.ca/1516/COURSE/course-MTE.html#MTE241}{Computer Structures \& Real-Time Systems\hfill 91\%}\par

\end{multicols}

\end{document}