% !TEX program = xelatex

\documentclass{resume}
\usepackage{comment}

%\usepackage{multicol}
%\begin{multicols}{2}
%\columnbreak
%\end{multicols}

\usepackage{hyperref} 
%\hypersetup{colorlinks=false}

%\usepackage{wrapfig}

\begin{document}
%\pagenumbering{gobble} % suppress displaying page number

\name{Oswin Rodrigues}
%\focus{Hardware}{Embedded}

% {E-mail}{mobilephone}{homepage}
\contactInfo{orodrigues@uwaterloo.ca}{+1-226-606-6220}{oswinrodrigues}

\section{\faWrench\ Tools}

\begin{itemize}[parsep=0.5ex]
  \item %Code: 
  C\textperiodcentered 
  C++\textperiodcentered
  Python\textperiodcentered
  Ladder Logic\textperiodcentered
  JavaScript\textperiodcentered
  MATLAB\textperiodcentered
  Assembly\textperiodcentered
  VHDL
  \item %Circuit:
  Circuit Design\textperiodcentered
  Arduino\textperiodcentered
  Soldering Iron\textperiodcentered
  Upverter\textperiodcentered
  EAGLE\textperiodcentered
  Oscilloscope\textperiodcentered
  Multimeter
  %\item %3D CAD Experience:
  %Solidworks\textperiodcentered
  %AutoCAD
\end{itemize}
\nspace{}

\section{\faGraduationCap\ Education}

\datedsubsection{\textbf{Mechatronics Engineering, Honors}, Candidate for BASc}{2013 -- Present}
University of Waterloo, Waterloo, ON \hfill Class of 2018\par 
\nspace{}

\section{\faLineChart\ Experience}

\datedsubsection{\textbf{EDA/CAD Engineer Intern}}{May -- Aug 2015}
\company{\textbf{Upverter Inc.}}{Toronto, ON}
\textit{Enhancing PCB CAD tool features in software (JavaScript, Python) and hardware avenues.}
\begin{itemize}
  \item Created and verified symbols and footprints for 150+ electronic components.
  \item Implemented component-tagging feature, using pin names to deduce functionality.
  \item Adjusted prioritization of design rules in layout constraint manager.
  \item Fixed incorrect drawing and positioning of constraint violation layout bodies.
  \item Refactored click event-listening logic in schematic net-drawing tool.
  \item Corrected book-keeping errors in pin manager for tracking connection mappings.
  \item Facilitated BGA footprint generator's omitting specific letters during row enumeration.
\end{itemize}
\nspacen{}

\datedsubsection{\textbf{Neuro-Robotics Lab Research Assistant}}{Feb -- Apr 2014}
\company{\textbf{University of Waterloo}}{Waterloo, ON}
\textit{Using ROS-run Turtlebot for social navigation research purposes.}
\begin{itemize}
  \item Wrote C++ and Python nodes to implement basic navigation stack on Turtlebot. 
  \item Published sensor, odometry and transform messages to mobile base.
  \item Tweaked existing open-source code for advanced algorithms: person-detection, SLAM navigation.
  %\item Troubleshot various problems arising due to incomplete-but-improving familiarity with ROS.
\end{itemize}
\nspacen{}

%\begin{comment}
\datedsubsection{\textbf{Mechanical Design Co-op}}{Sep -- Dec 2014}
\company{\textbf{Prodomax Automation Inc.}}{Barrie, ON}
\textit{Designing jigs and fixtures in Solidworks for automotive part-assembly stations.}
\begin{itemize}
  \item Modeled custom tooling in two stations for assembling a vehicle's seat track mechanism. 
  \item One station inserted an anti-collapse spacer and the other stamp-pressed a bushing.
  \item Detailed and ballooned numerous part and assembly drawings extensively.
\end{itemize}
%\end{comment}
\nspace{}

\section{\faFlask\ Projects}

\datedsubsection{\textbf{UW Robotics Team \& WAVE\textsuperscript{1}
%\footnote{Waterloo Autonomous Vehicles}\
 Lab}}{Jan -- Apr 2015, Sep 2015 -- Present}
\textsuperscript{1}\textit{Waterloo Autonomous Vehicles}
\begin{itemize}
  \item Modified EAGLE schematics and layouts for Arduino motor shield. 
  \item Soldered different SMT and THT components onto multiple bare shields.
  \item Currently designing and implementing a wireless (RF) e-stop mechanism for racing robot.
  \item Currently revamping and parts-sourcing a Mars Rover's electrical box. 
\end{itemize}
\nspacen{}

\datedsubsection{\href{https://github.com/oswinrodrigues/LED-Matrix-And-Accelerometer}{\textbf{Tilt-Sensitive LED Matrix Panel}}}{Personal Project, Ongoing}
\textit{`Moving' a single lit LED on the panel by physically tilting it. This uses:}
\begin{itemize}
  \item Arduino microcontroller for handling the `smarts and magic'.
  \item ADXL335 accelerometer for controlling the tilt functionality. 
  \item 74HC595N shift register (SIPO) for I/O expansion on the Arduino board.
\end{itemize}
\nspacen{}

\datedsubsection{\textbf{Hackathons}}{Various}
\begin{itemize}
  \item \href{https://github.com/DChang87/HtN}{Pebble-facilitated dosage notification service - \textit{SmartMeds}; used C.} \hfill Hack the North, 2015
  \item \href{jkkd.com}{IMU-based instructor - \textit{Yoga Yoda}; developed business case. \hfill} PCH Hardware Hackathon, 2015
  \item \href{http://devpost.com/software/drumyo}{Myo-controlled air drum kit - \textit{DruMyo}; used C++. \hfill} hackWaterloo, 2014
  \item \href{https://github.com/fanwashere/SolidWorksControls}{Myo-enabled Solidworks controller; used Lua. \hfill} Hack the North, 2014
\end{itemize}

\begin{comment}
\datedsubsection{\textbf{Open-Ended Course\textsuperscript{2} Project}}{Sep -- Dec 2013}
\textsuperscript{2}\textit{Mechatronics Engineering, Digital Computation}
\begin{itemize}
  \item Built a fully functional robot on the Lego Mindstorms NXT platform.
  \item 
\end{itemize}
\end{comment}
\nspace{}

\section{\faBook\ Courses}
\begin{itemize}
\item \href{http://www.ucalendar.uwaterloo.ca/1516/COURSE/course-MTE.html#MTE120}{Circuits \hfill 93\%} \par
\item \href{http://www.ucalendar.uwaterloo.ca/1516/COURSE/course-MTE.html#MTE220}{Sensors \& Instrumentation \hfill ---}\par
\item \href{http://www.ucalendar.uwaterloo.ca/1516/COURSE/course-MTE.html#MTE140}{Data Structures \& Algorithms \hfill 94\%} \par
\item \href{http://www.ucalendar.uwaterloo.ca/1516/COURSE/course-MTE.html#MTE262}{Microprocessors \& Digital Logic \hfill 78\%}\par
\item \href{http://www.ucalendar.uwaterloo.ca/1516/COURSE/course-MTE.html#MTE241}{Computer Structures \& Real-Time Systems \hfill ---}
\end{itemize}
\nspace{}

\section{\faPaintBrush\ Non-Technical}
\begin{itemize}
\item East Coast Swing
\item Guitar \& Drums
\item Discovery Trips
\item \textit{Naruto Shippuden}
\item Rap \& Poetry
\item Basketball
\end{itemize}
\begin{comment}
\begin{itemize}
  \item \textbf{East Coast Swing}:\par 
  ``It don't mean a thing if it ain't got that swing" - Irving Mills. This dance form makes it worth getting your two left feet right.
  \item \textbf{Guitar and Drums}:\par
  Whether it's playing in front of scores at church or a handful around a bonfire, the jamming experience is truly revitalizing.
  \item \textbf{Discovery Outings}:\par 
  Curiosity outdoors gets the best of me, even when it's `just' another trail that I've spotted. Adventure always awaits.
  \item \textit{\textbf{Naruto Shippuden}}:\par 
  The sheer number of life lessons I've learned from this anime alone is phenomenal. I remain unashamed and a child at heart forever.
  \item \textbf{Rap and Poetry}:\par 
  Words. They're mighty weighty. And the potential for good art to flow from them is why I've come to appreciate and experiment with rap.
\end{itemize}
\end{comment}
\nspace{}

\end{document}